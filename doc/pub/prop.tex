\documentclass[a4paper]{article}

\usepackage[utf8]{inputenc}
\usepackage[T1]{fontenc}
\usepackage{textcomp}

\usepackage{amsmath}
\usepackage{amssymb}
\usepackage{graphicx}

\title{Feature Evaluation for Artwork Image Retrieval}
\author{Michal Grochmal
  $<$\href{mailto:grochmal@member.fsf.org}{grochmal@member.fsf.org}$>$
}
\date{\today}

\usepackage[pdftex,colorlinks=true]{hyperref}

\begin{document}
\maketitle

\section{Introduction}

Content Based Information Retrieval systems for images or simply Content Based
Image Retrieval systems (CBIR systems) retrieve from large collections images
similar to a query image.  Yet, what "similar" means in this context vary,
similarity depends on the nature of the collection and on what meaning
(semantic) contained in the query image we want to search for.  A specific
query image may contain more than one meaning.  e.g. we might query a
collection of personal photos using a picture of our cousin holding a carnival
mask in Venice.  For this query image we might want to ask: whether there are
photos of our cousin in the collection, or whether there are photos of venetian
carnival masks, or photos with a specific building or location in Venice in the
background.

Each type of collection of images needs a different set of features that allows
searching for similar images within a semantic context.  In this work we will
evaluate features for CBIR in the retrieval of art images in the contexts of
similar style of illustrations, drawings and paintings.  Aesthetics are deeply
related to the concept of art \cite{rmc12ajs} therefore similar style will be
discussed by the evaluation of aesthetics of the images, while ignoring the
content of the image.  Back to the example of the query image with our cousin
holding a carnival mask in Venice we would try to evaluate the quality of the
framing of the photo, and ignoring the presence of our cousin, the mask and the
location.

Although completely ignoring the content of an image is probably impossible,
features that are independent of the objects in the image exist.  Such features
are more relevant in works of art \cite{zirnhelt07art} as the meaning of how
objects are drawn is important for how humans see the piece of art
\cite{mach10clas}.  In this work we will use aesthetic features tried on
collections of drawn artwork (illustration, drawing and painting) and on
collections of photography and apply them for retrieval of artwork with the
meaning of identifying the author of the artwork.  Further, we will try to
extrapolate the meaning of these features and try to identify and retrieve
images of the same art school and same art period.

\section{Current work}

Production CBIR systems for art, two of which are described in \cite{cfsp12air}
and in \cite{ymvz03tree}, use colour, colour disposition and texture as
features to retrieve drawings, illustrations and/or paintings.  These features
capture the style of the image better than corner detection or interest points
(as in SIFT or MSER) used in object recognition and duplicate detection
\cite{szel11book}, as the former describe the content of the image.

On research ground works explore richer sets of features, yet these works focus
on photography rather than drawn art.  Estimates of image complexity based on
lossy compression errors, standard deviation and average over components of HSV
representation are used by Romeo and Correia in
\cite{jma12clas,cmrc13fs,rmc12ajs}.  Itten colour harmony over components of
the HSL representation and line estimates drawn from edges detected in the
image are used in \cite{mach10clas}.  All these features are added to colour
and texture to produce features for image classification.  Ivanova
\cite{isv12mpeg} uses MPEG-7 descriptors as features, yet MPEG-7 descriptors
are just the same features encoded in a different way.  The MPEG-7 features
used are: colour histograms, colour distribution, dominant colour, edge
histogram, and texture.  We can argue that the features explored are consistent
among all these works.

A closely related field to CBIR in art is Affective Image Retrieval (AIR)
summed by Machajdik in \cite{mach10ua,mach10clas}.  In this field both the
style and the content of the image is important, yet it is important that the
style and the content are separately evaluated.  Although in the project at
hand we are not interested in the content of the images the features used in
AIR to capture the style of the image may prove useful.  In AIR, features from
art theory based on Itten colours can classify an image into emotions.  Itten
colours deal with colour harmony and warmth or coldness of colours.  Also,
Itten colours are well defined in mathematical terms which make them image
features that are simple to compute.

All features above are scale dependent \cite{rmc12ajs,mach10clas} therefore
normalisation of the size and colour space of the image must be performed.  The
HSV/HSL colour spaces are used instead of RGB because features extracted from
those colour spaces can be linked to human perception.  Even better, we can
link a feature extracted from only one component of HSV or HSL to human
perception (e.g. how bright the image is), this is not possible in the RGB
colour space \cite{rmc12ajs}.

The major datasets used in \cite{jma12clas,mach10clas} are of photography
(because of the wide availability of such datasets) therefore some of the
features have not yet been tried on drawn art.  On the other hand small
datasets of art were tried in \cite{mach10clas} and in \cite{rmc12ajs};  also,
in \cite{rmc12ajs} a task of author identification was successful.  A bigger
dataset of drawn art was tried in \cite{zirnhelt07art} but using a very limited
set of features.  Finally, a task of artistic period and art school
identification is proposed in \cite{zirnhelt07art} and \cite{rmc12ajs} based on
the features used in the author identification task.  In this work we plan to
fill the gap of applying the features that have not yet been tried on drawn art
to a dataset of this kind of images.

\section{Proposed methodology}

Although it is not possible to fully separate the content of the image from
it's style, there are good approximations by using features general to the
image.  We will reproduce the features used by Romeo in \cite{rmc12ajs} and by
Machajdik in \cite{mach10clas} and apply these features to the task of author,
period and art school classification.  To perform this task we will need
software for programmatic image processing and for classification tasks.

The OpenCV\footnote{available at \href{http://opencv.org/}{opencv.org} under
the BSD license} library provides a big set of functions that can be used to
extract features from images, and it is well integrated with many programming
languages.  Yet, OpenCV is not very good at quickly prototyping because it must
be called as a library from a programming language.  For prototyping the
feature extraction the ImageMagick\footnote{available at
\href{http://www.imagemagick.org/}{www.imagemagick.org} under the Apache 2.0
license} set of command line tools will be used.  For a third option, if one of
the tools miss needed functionality, the python imaging library
(PIL)\footnote{available at
\href{http://pythonware.com/products/pil/}{pythonware.com/products/pil} under a
generic open source license} (or it's fork Pillow) can be used as described by
Solem in \cite{solem12book}.  The most complex image transformations for
feature extraction will be the Sobel and Canny filters for edge detection,
these techniques are available from OpenCV or can be simulated with PIL and
scipy \cite{oliphant06numpy}.

For the classification task Support Vector Machines (SVMs) will be used, as
this technique was previously used for the author identification task by Romeo
in \cite{rmc12ajs}.  SVMs are available from OpenCV.  Yet, for prototyping,
LIBSVM\footnote{available at \href{http://www.csie.ntu.edu.tw/~cjlin/libsvm/}
{www.csie.ntu.edu.tw/~cjlin/libsvm} under a generic open source license}
command line tools will be used, as described in \cite{hcl03svm}.  This tool
set shall be enough to replicate previous experiments and to extrapolate the
experiments onto new datasets.

\subsection{Datasets}

Comparison between works can only be achieved if the datasets used for the
classification are published for the research community \cite{mach10clas}.  And
the datasets used in \cite{mach10clas} and in \cite{jma12clas} are published.
Yet, these two datasets are mostly of photographs, whilst the artist
identification task is to be performed over drawn art.  These two datasets are
useful to test to which level the work can be replicated and then it needs to
be applied over new, more suitable, datasets.

Online art collections are available from galleries and museums.  One such
collection originating from the Birkbeck College\footnote{
\href{http://www.bbk.ac.uk}{bbk.ac.uk}} through the CACHe project\footnote{
\href{http://www.bbk.ac.uk/hosted/cache/}{www.bbk.ac.uk/hosted/cache}} is the
NICE paintings (NIRP) collection.  This collection is currently available from
the Visual Arts Data Service (VADS)\footnote{
\href{http://vads.ac.uk/}{vads.ac.uk}} project, and is tagged with metadata
about the author and period of the items.  Another suitable dataset is the
collection of the Victoria and Albert Museum (VAM)\footnote{
\href{http://www.vam.ac.uk/}{www.vam.ac.uk}} and it's vast API leading to well
described art items in it's collections.  Both NIRP and VAM collections use the
Dublin Core Metadata Initiative\footnote{
\href{http://dublincore.org/}{dublincore.org}} as a standard description form
for the items in the collections, therefore the items from both collections can
be identified in the same way.  The use of a careful selection of items from
the NIRP and VAM collections as datasets allows for space to extrapolate the
features proposed for author identification.

\subsection{Milestones}

First of all, we will need to replicate the work by Romeo in \cite{rmc12ajs}.
The major task for this replication is the construction of the estimate of the
Koglomorov complexity by JPEG and fractal image compression, as described in
Romeo's work. Once the replication of the features used by Romeo is complete we
will add some of the features described by Machajdik in \cite{mach10clas}.  In
the second set of features the majority of work will be in the implementation
of features related to Itten colours.

We will crawl the NIRP and VAM collections for illustrations, drawings and
paintings with complete metadata.  The VAM collection contains photographies of
sculptures, porcelain and/or other non-drawn art and the NIRP collection
contains items with missing metadata; such images would add extra complexity
not envisioned for the classification task and will be removed from the
datasets.  The resulting datasets will contain only paintings, illustrations
and drawings classified by the author and period of the item.

We will extrapolate the feature extraction and classification procedure
previously replicated to the datasets from NIRP and VAM collections.  For each
task of author identification and school/period identification the datasets
will be tried as two separate dataset and as one combined dataset.  In each
case the dataset will be divided into training and testing sets and evaluated.

\subsection{Testing}

Replication of Romeo's work and of features from Machajdik's work shall be
evaluated towards achieving similar performance on the same dataset.  The
performance measure used by Romeo in \cite{rmc12ajs} is the percentage of
correctly classified images for each class (or author).  The dataset from
Romeo's work is published and be used in a classification task that shall
replicate the previous performance.  With the addition of features from
\cite{mach10clas} the classification performance shall not deplete.  If the
performance of the classifier depletes with the new features these shall be
removed.

Author identification task over the NIRP and VAM datasets will be then
performed using the same features and measured in the same way.  Romeo's work
on author identification contained only 6 authors (Goya, Monet, Gauguin, Van
Gogh, Kandinsky and Picasso) and from a limited time period (19th and early
20th century), yet NIRP and VAM collections contain items from many more
authors and periods.  This difference may produce different results, proving
the set of features good or insufficient for author identification in a more
heterogeneous collection.

%Fallback

If the features prove insufficient we will perform Principal Component Analysis
(PCA) of the NIRP and VAM datasets against the dataset used in Romeo's work
using the features as components.  The difference of the disposition of
features in the datasets shall be visible in the PCA.

Finally, if time permits, we will extrapolate the author identification task
towards art period/school identification.  Although this task is not much
different from author identification the extra complexity lies in the fact that
different collections are not consistent about this classification.  As argued
by DiMaggio in \cite{dimaggio87art} classification of art cannot be unified as
it is part of the behaviour of social groups, which always changes.  Therefore
the art period classification will need to allow an item to be classified into
many classes at once.  And the classification of items will use different
classes depending on the collection the it is made against.

\subsection{Limitations and out of scope issues}

One of the most important features proposed is the colour histogram.  The
reason for this is that before the second half of the 20th century artists made
their pigments themselves.  Yet in the second half of the 20th century
industrialised paints standardised colours and art from this period is harder
to distinguish by this feature.  Moreover, with digital enhancements to
illustrations starting from the '90s of the previous century, the entire colour
spectrum was made available to all artists making the colour an even less
important feature for these images.  Classifying art from this period using the
same features will probably render worse results, this issue is not covered in
the project.

Other flat art items (flat sculpture, porcelain or some forms of modern art)
could be classified in similar fashion as illustration, drawing and painting.
Yet, here we do not consider any of the complications that might arise from
classifying this kind of art using the same features.  Such art items are out
of the scope of this project.

\bibliographystyle{plain}
\bibliography{capybara}

\end{document}

emotional response to colour [mach10clas]

Using dataset from VADS, VA museus, BBK collections, CACHe, (ref to CHARt)

No architecture or sculpture, the photo of it have style in itself.

